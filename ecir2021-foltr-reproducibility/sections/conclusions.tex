\section{Conclusions}

We set to explore 4 research questions. RQ1 aimed to investigate the generalisability of the original results obtained  by FOLTR-ES on the MQ2007/2008 dataset to other dataset used in current OLTR practice. Our reproducing experiments on MQ2007/2008 show the consistently similar findings with the original work. However, in terms of larger LTR datasets (MSLR-WEB10K and Yahoo datasets), the neural ranker is less effective than the linear ranker, especially on MSLR-WEB10K, which shows FOLtR-ES needs more data to achieve effective training on those datasets with larger number of features.
RQ2 aimed to investigate the effect varying the number of clients involved in FOLTR-ES has on the effectiveness of the method. \todo{Our findings suggest that ...}
RQ3 aimed to compare FOLRT-ES with current OLTR state-of-the-art methods to understand the gap required to be paid for maintaining privacy. \todo{Our findings suggest that ...}
RQ4 aimed to investigate the generalisability of the original results obtained for FOLTR-ES to common evaluation practice in OLTR. \todo{Our findings suggest that ...}