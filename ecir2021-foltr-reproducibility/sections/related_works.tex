\section{Related Work}

This section gives a brief overview of traditional LTR, OLTR and federated learning. 

\subsection{Offline Learning to Rank}
The core of Learning to Rank (LTR) is to train a ranking model using Machine Learning methods.

\subsection{Online Learning to Rank}
Main challenges in online learning to rank problem are: (a) noise and biases of user interactions, (b) non-continuous metrics for optimization.

OLTR updates the ranker through user's online feedback, more specifically, user's click data. Unlike traditional LTR methods, OLTR doesn't need human-annotated data, which is expensive and time-consuming to create. Besides, human-annotated data can not always represent true preference of users. Traditional LTR methods are more suitable in situation where one ranking list fits all users. In a continuously developing world, people's preference and answers to one search query may change or differ from other's. OLTR provides a better way to learning from users' feedback in order to improve users' experience and ranking effectiveness.

\subsection{Federated Machine Learning}

Federated Machine Learning aims to solve two main challenges existing in today's artificial intelligence research: (a) data existing in the form of isolated islands, (b) the growing concern for data privacy and security~\cite{yang2019federated}. 

The concept of Federated Learning was first proposed by Google in 2016~\cite{DBLP:journals/corr/KonecnyMRR16,DBLP:journals/corr/KonecnyMYRSB16}. The first Federated Learning algorithm is Federated Averaging~\cite{mcmahan2016federated}. In early year's research, the Federated Learning aims to train Machine Learning models using data from multiple data source while preventing data leakage. 
