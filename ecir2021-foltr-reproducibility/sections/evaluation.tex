\section{Experimental Settings}



1. datasets\\
The datasets we use are MQ2007 and MQ2008[1], MLSR-WEB10K[1], Yahoo! Webscope[2] and Istella[3] datasets.\\
MQ2007 dataset contains 1692 unique queries and 69623 relevance labels. MQ2008 dataset contains 784 unique queries and 15211 relevance labels. For MQ2007 and MQ2008 dataset, each query-document pair contains 46 features.\\
Each dataset is provided with five splits, except for ...\\
We use the original dataset in training and validating procedure without any preprocessing, such as feature normalization.\\
2. evaluation metric\\
3. baselines\\
4. models\\
5. click models: we simulate uses by applying an instance of Position Biased Model (PBM) [x]\\
(further explain about PBM)\\
table for PBM we use\\

[1]Tao Qin and Tie-Yan Liu. 2013. Introducing LETOR 4.0 datasets. arXiv:1306.2597 (2013)
[2]Chapelle, O., Chang, Y.: Yahoo! learning to rank challenge overview. J. Mach. Learn. Res. 14, 1–24 (2011)
[3]Dato, D., Lucchese, C., Nardini, F.M., Orlando, S., Perego, R., Tonellotto, N., Venturini, R.: Fast ranking with additive ensembles of oblivious and non-oblivious regression trees. ACM Trans. Inform. Syst. (TOIS), 35(2) (2016). Article 15

@article{DBLP:journals/corr/QinL13,
	author    = {Tao Qin and
		Tie{-}Yan Liu},
	title     = {Introducing {LETOR} 4.0 Datasets},
	journal   = {CoRR},
	volume    = {abs/1306.2597},
	year      = {2013},
	url       = {http://arxiv.org/abs/1306.2597},
	timestamp = {Mon, 01 Jul 2013 20:31:25 +0200},
	biburl    = {http://dblp.uni-trier.de/rec/bib/journals/corr/QinL13},
	bibsource = {dblp computer science bibliography, http://dblp.org}
}