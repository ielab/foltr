\documentclass[runningheads]{llncs}
\usepackage{graphbox,graphicx}
\usepackage[labelfont=bf]{caption}
\usepackage{color}
\usepackage{subcaption}
\usepackage{booktabs}
\usepackage{multirow}
\usepackage[utf8]{inputenc}
%\usepackage{fourier} 
\usepackage{array}
\usepackage{makecell}
\usepackage{listings}
\usepackage{hyperref}
\usepackage{scalerel}
\usepackage{float}
\usepackage{amsmath, amssymb, calrsfs, wasysym, verbatim, bbm, color, graphics, geometry} 
\usepackage{bm}
\usepackage{cite}
%\usepackage{subfigure}
\graphicspath{ {./images/} }
\usepackage{tabularx}
\usepackage{enumitem}
\setlist[itemize]{leftmargin=40pt}
\usepackage{subcaption}
\usepackage[misc]{ifsym}

\renewcommand\UrlFont{\color{blue}\rmfamily}
\newcommand{\todo}[1]{}
\renewcommand{\todo}[1]{{\color{red}{#1}}}

\newcommand{\customsymbol}[1]{\scalerel*{\includegraphics{#1}}{`}}
\begin{document}

\title{Federated Online Learning to Rank with Evolution Strategies: A Reproducibility Study}

\author{Shuyi Wang\textsuperscript{(\Letter)}\orcidID{0000-0002-4467-5574}
	\and 
	Shengyao Zhuang\orcidID{0000-0002-6711-0955}
	\and 
	Guido Zuccon\orcidID{0000-0003-0271-5563}}
\authorrunning{S. Wang et al.}

\institute{The University of Queensland, St Lucia, Australia\\ \email{\{shuyi.wang, s.zhuang, g.zuccon\}@uq.edu.au}}

\maketitle

\begin{abstract}
Online Learning to Rank (OLTR) optimizes ranking models using implicit users' feedback, such as clicks, directly manipulating search engine results in production. This process requires OLTR methods to collect user queries and clicks; current methods are not suited to situations in which users want to maintain their privacy, i.e. not sharing data, queries and clicks. 

Recently, the federated OLTR with evolution strategies (FOLtR-ES) method has been proposed to provide a solution that can meet a number of users' privacy requirements. Specifically, this method exploits the federated learning framework and $\epsilon$-local differential privacy. However, the original research study that introduced this method only evaluated it on a small Learning to Rank (LTR) dataset and with no conformity with respect to current OLTR evaluation practice. It further did not explore specific parameters of the method, such as the number of clients involved in the federated learning process, and did not compare FOLtR-ES with the current state-of-the-art OLTR method. This paper aims to remedy to this gap. 

Our findings question whether FOLtR-ES is a mature method that can be considered in practice: its effectiveness largely varies across datasets, click types, ranker types and settings. Its performance is also far from that of current state-of-the-art OLTR, questioning whether the maintained of privacy guaranteed by FOLtR-ES is not achieved by seriously undermining search effectiveness and user experience. 

\keywords{Online learning to rank \and Federated machine learning \and Differential privacy}
\end{abstract}

\section{Introduction}

one sentence for federated learning

The following research questions are addressed in this paper:

RQ1. Does the Federated Online Learning to Rank algorithm also work on other widely used Learning to Rank datasets?
RQ2. Does the number of clients affect the performance of Federated Online Learning to Rank?
RQ3. Can federated online learning to rank achieve similar performance comparing with other state-of-the-art online learning to rank methods?
RQ4. Can Federated Online Learning to Rank generalise to other typical OLTR evaluation setup, such as nDCG or MRR?


\section{Federated OLTR with Evolution Strategies}\label{sec:method}

We provide a brief overview of the FOLtR-ES method, which extends online LTR to federated learning; this is done by exploiting evolution strategies optimization, a widely used paradigm in Reinforcement Learning. 
The FOLtR-ES method consists of three parts. First, it casts the ranking problem into the federated learning optimization setting. Second, it uses evolution strategies to estimate gradients of the rankers. Finally, it introduces a privatization procedure to further protect users' privacy.

\subsection{Federated Learning Optimization Setting}
The federated learning optimization setting consists in turn of several steps, and assumes the presence of a central server and a number of distributed clients. First, a client downloads the most recently updated ranker from the server. Afterwards, the client observes $B$ user interactions (search queries and examination of SERPs) which are served by the client's ranker. The performance metrics of these interactions are averaged by the client and a privatized message is sent to the centralized server. After receiving messages from $N$ clients, the server combines them to estimate a single gradient $g$ and performs an optimization step to update the current ranker. Finally, the clients download the newly updated ranker from the server.

\subsection{Gradient Estimation} \label{sec-gradient-est}
The method assumes that the ranker comes from a parametric family indexed by vector $\theta \in R^{n}$. Each time a user $u$ has an interaction $a$, the ranking quality is measured; this is denoted as $f$. The goal of optimization is to find the vector $\theta^*$ that can maximize the mean of the metric $f$ across all interactions $a$ from all users $u$:
\begin{equation}
	\theta^{*}=\arg \max _{\theta} F(\theta)=\arg \max _{\theta} \mathbb{E}_{u} \mathbb{E}_{a \mid u, \theta} f(a ; \theta, u) \label{eq-theta}
\end{equation}

Using Evolution Strategies (ES)~\cite{salimans2017evolution}, FOLtR-ES considers a population of parameter vectors which follow the distribution with a density function $p_{\phi}(\theta)$. The objective aims to find the distribution parameter $\phi$ that can maximize the expectation of the metric across the population:
\begin{equation}
	 \mathbb{E}_{\theta\sim p_{\phi}(\theta)}~[F(\theta)] \label{eq-expectation}
\end{equation}

The gradient $g$ of the expectation of the metric across the population (Equation~\ref{eq-expectation}) is obtained in a manner similar to REINFORCE~\cite{williams1992simple}:
\begin{equation}
	\begin{aligned}
		g &=\nabla_{\phi} \mathbb{E}_{\theta}[F(\theta)]=\nabla_{\phi} \int_{\theta} p_{\phi}(\theta) F(\theta) d \theta=\int_{\theta} F(\theta) \nabla_{\phi} p_{\phi}(\theta) d \theta=\\
		&=\int_{\theta} F(\theta) p_{\phi}(\theta)\left(\nabla_{\phi} \log p_{\phi}(\theta)\right) d \theta=\mathbb{E}_{\theta}\left[F(\theta) \cdot \nabla_{\phi} \log p_{\phi}(\theta)\right]
	\end{aligned}
\end{equation}

Following the Evolution Strategies method, FOLtR-ES instantiates the population distribution $p_{\phi}(\theta)$ as an isotropic multivariate Gaussian distribution with mean $\phi$ and fixed diagonal covariance matrix $\sigma^2I$. Thus a simple form of gradient estimation is denoted as:
\begin{equation}
	g=\mathbb{E}_{\theta \sim p_{\phi}(\theta)}\left[F(\theta) \cdot \frac{1}{\sigma^{2}}(\theta-\phi)\right]
\end{equation}

Based on the federated learning optimization setting, $\theta$ is sampled independently on the client side. Combined with the definition of $F(\theta)$ in Equation~\ref{eq-theta}, the gradient can be obtained as:
\begin{equation}
	g=\mathbb{E}_{u} \mathbb{E}_{\theta \sim p_{\phi}(\theta)}\left[\left(\mathbb{E}_{a \mid u, \theta} f(a ; \theta, u)\right) \cdot \frac{1}{\sigma^{2}}(\theta-\phi)\right] \label{eq-gradient}
\end{equation}

To obtain the estimate $\hat{g}$ of $g$ from Equation~\ref{eq-gradient}, $\hat{g} \approx g$, the following steps are followed: (i) each client $u$ randomly generates a pseudo-random seed $s$ and uses the seed to sample a perturbed model $\theta_{s} \sim \mathbb{N}\left(\phi, \sigma^{2} I\right)$, (ii) the average of metric $f$ over $B$ interactions is used to estimate the expected loss $\hat{f} \approx \mathbb{E}_{a \mid u, \theta_{s}} f(a;\theta_s, u) $ from Equation~\ref{eq-gradient}, (iii) each client communicates the message tuple $(s,\hat{f})$ to the server, (iv) the centralized server computes the estimate $\hat{g}$ of Equation~\ref{eq-gradient} according to all message sent from the $N$ clients.

To reduce the variance of the gradient estimates, means of antithetic variates are used in FOLtR-ES: this is a common ES trick~\cite{salimans2017evolution}. The algorithm of the gradient estimation follows the standard ES practice, except that the random seeds are sampled at the client side.

\subsection{Privatization Procedure}
To ensure that the clients' privacy is fully protected, in addition to the federated learning setting, FOLtR-ES also proposes a privatization procedure that introduces privatization noise in the communication between the clients and the server.

Assume that the metric used on the client side is discrete or can be discretized if continuous. Then, the metric takes a finite number ($n$) of values, $f_0, f_1, ..., f_{n-1}$. For each time the client experiences an interaction, the true value of the metric is denoted as $f_0$ and the remaining $n-1$ values are different from $f_0$. When the privatization procedure is used, the true metric value $f_0$ is sent with probability $p$. Otherwise, with probability $1-p$, a randomly selected value $\hat{f}$ out of the remaining $n-1$ values is sent. To ensure the same optimization goal described in Section~\ref{sec-gradient-est}, FOLtR-ES assumes that the probability $p > 1/n$.

Unlike other federated learning methods, FOLtR-ES adopts a strict notion of $\epsilon$-local differential privacy~\cite{kharitonov2019federated}, in which the privacy is considered at the level of the client, rather than of the server. Through the privatization procedure, $\epsilon$-local differential privacy is achieved, and the upper bound of $\epsilon$ is:
\begin{equation}
	\epsilon \leq log\frac{p(n-1)}{1-p} 
\end{equation}

This means that, thanks to the privatization scheme, at least $log[p(m-1)/(1-p)]$-local differential privacy can be guaranteed. At the same time, any $\epsilon$-local differential private mechanism also can obtain $\epsilon$-differential privacy~\cite{dwork2014algorithmic}.
\section{Experimental Settings}



1. datasets\\
The datasets we use are MQ2007 and MQ2008[1], MLSR-WEB10K[1], Yahoo! Webscope[2] and Istella[3] datasets.\\
MQ2007 dataset contains 1692 unique queries and 69623 relevance labels. MQ2008 dataset contains 784 unique queries and 15211 relevance labels. For MQ2007 and MQ2008 dataset, each query-document pair contains 46 features.\\
Each dataset is provided with five splits, except for ...\\
We use the original dataset in training and validating procedure without any preprocessing, such as feature normalization.\\
2. evaluation metric\\
3. baselines\\
4. models\\
5. click models: we simulate uses by applying an instance of Position Biased Model (PBM) [x]\\
(further explain about PBM)\\
table for PBM we use\\

[1]Tao Qin and Tie-Yan Liu. 2013. Introducing LETOR 4.0 datasets. arXiv:1306.2597 (2013)
[2]Chapelle, O., Chang, Y.: Yahoo! learning to rank challenge overview. J. Mach. Learn. Res. 14, 1–24 (2011)
[3]Dato, D., Lucchese, C., Nardini, F.M., Orlando, S., Perego, R., Tonellotto, N., Venturini, R.: Fast ranking with additive ensembles of oblivious and non-oblivious regression trees. ACM Trans. Inform. Syst. (TOIS), 35(2) (2016). Article 15

@article{DBLP:journals/corr/QinL13,
	author    = {Tao Qin and
		Tie{-}Yan Liu},
	title     = {Introducing {LETOR} 4.0 Datasets},
	journal   = {CoRR},
	volume    = {abs/1306.2597},
	year      = {2013},
	url       = {http://arxiv.org/abs/1306.2597},
	timestamp = {Mon, 01 Jul 2013 20:31:25 +0200},
	biburl    = {http://dblp.uni-trier.de/rec/bib/journals/corr/QinL13},
	bibsource = {dblp computer science bibliography, http://dblp.org}
}
\section{Results and Analysis}


\subsection{RQ1: Generalisation of FOLTR-ES performance beyond MQ2007/2008}
For answering RQ1 we replicate the results obtained by Kharitonov~\cite{kharitonov2019federated} on the MQ2007 and MQ2008 datasets; we then reproduce the experiment on MSLR-WEB10K and Yahoo datasets, on which FOLTR-ES has not been yet investigated, and we compare the findings across datasets. For these experiments we use antithetic variates, set $B = 4$ and 2,000 clients, use MaxRR as reward signal and for evaluation on clicked items. 

\begin{figure}[t]
	\centering
	\begin{subfigure}{1\textwidth}
		\includegraphics[width=15cm, height=3.5cm]{images/RQ1/mq2007_foltr_c2000_ps.png}
		\caption{Mean batch MaxRR for MQ2007.}
		\label{fig:mq2007-rq1}
	\end{subfigure}
	\begin{subfigure}{1\textwidth}
		\includegraphics[width=15cm, height=3.5cm]{images/RQ1/mslr10k_foltr_c2000_ps.png}
		\caption{Mean batch MaxRR for MSLR10k}
		\label{fig:mslr10k-rq1}
	\end{subfigure}
	\begin{subfigure}{1\textwidth}
		\includegraphics[width=15cm, height=3.5cm]{images/RQ1/yahoo_foltr_c2000_ps.png}
		\caption{Mean batch MaxRR for Yahoo.}
		\label{fig:yahoo-rq1}
	\end{subfigure}
	\caption{Results for RQ1: performance of FOLTR-ES across datasets. \label{fig:RQ1}}
\end{figure}

Figure~\ref{fig:mq2007-rq1} reports the results obtained by FOLTR-ES on the MQ 2007 dataset\footnote{Similar results were obtained for MQ 2008 and are omitted for space reasons.} with respect to the three click models considered, various settings for the privacy preservation parameter $p$, and the two FOLTR-ES methods (linear and neural). Our results fully replicate those of Kharitonov~\cite{kharitonov2019federated} and indicate the following findings: (1) FOLTR-ES allows for the iterative learning of effective rankers; (2) high values of $p$ (lesser privacy) provide higher effectiveness; 
(3) the neural ranker is more effective than the linear ranker when $p \rightarrow 1$ (small to no privacy), while the linear model is equivalent, or better (for informational clicks) when $p=0.5$. 

However, not all these findings are applicable to the results obtained when considering MSLR-WEB10K and Yahoo, which are displayed in Figures~\ref{fig:mslr10k-rq1} and~\ref{fig:yahoo-rq1}. In particular, we observe that (1) the results for MSLR-WEB10K (and to a lesser extent also for Yahoo) obtained with the informational click model are very unstable, and, regardless of the click model, FOLTR-ES requires more data than with MQ 2007/2008 to arrive at a stable performance, when it does; (2) the neural ranker is less effective than the linear ranker, especially on MSLR-WEB10K. We believe these findings are due to the fact that query-document pairs in MSLR-WEB10K and Yahoo are represented by a larger number of features than in MQ2007/2008. Thus, more data is required for effective training, especially for the neural model; we also note that FOLTR-ES is largely affected by noisy clicks in MSLR-WEB10K. 

%In the previous work, FOLtR-ES is conducted on MQ2007 and MQ2008 datasets~\cite{kharitonov2019federated}. To further study if the algorithm can achieve similar ranking performance on other publicly available LTR datasets. We perform experiments on MSLR-WEB10K dataset using same parameters chosen by~\cite{kharitonov2019federated}, using antithetic variates and setting $B = 4$. 

%Unlike MQ2007 and MQ2008 datasets, FOLtR-ES performed on MSLR-WEB10K shows an opposite finding: the neural ranker dose not consistently perform better that the linear ranker. And for MSLR-WEB10K, FOLtR-ES takes more times on updating the ranker till it achieves the stable performance, which might be caused by larger training queries in MSLR-WEB10K. Figure \ref{fig: mq2007-rq1-1.0}\ref{fig: mslr-rq1-1.0}\ref{fig: mq2007-rq1-0.5}\ref{fig: mslr-rq1-0.5} show the mean batch MaxRR averaged on five data splits in MQ2007 and MSLR-WEB10K with the three click models.




%(Figures lack legend)
%\begin{figure}[H]
%	\centering
%	\includegraphics[width=15cm, height=3.5cm]{mq2007_foltr_c2000_p1.0.png}
%	\caption{Mean batch MaxRR for MQ2007 (2000 clients and $p = 0.9$)}
%	\label{fig: mq2007-rq1-1.0}
%\end{figure}
%\begin{figure}[H]
%	\centering
%	\includegraphics[width=15cm, height=3.5cm]{mslr10k_foltr_c2000_p1.0.png}
%	\caption{Mean batch MaxRR for MSLR-WEB10K (2000 clients and $p = 0.9$)}
%	\label{fig: mslr-rq1-1.0}
%\end{figure}
%
%\begin{figure}[H]
%	\centering
%	\includegraphics[width=15cm, height=3.5cm]{mq2007_foltr_c2000_p0.5.png}
%	\caption{Mean batch MaxRR for MQ2007 (2000 clients and $p = 0.5$)}
%	\label{fig: mq2007-rq1-0.5}
%\end{figure}
%\begin{figure}[H]
%	\centering
%	\includegraphics[width=15cm, height=3.5cm]{mslr10k_foltr_c2000_p0.5.png}
%	\caption{Mean batch MaxRR for MSLR-WEB10K (2000 clients and $p = 0.5$)}
%	\label{fig: mslr-rq1-0.5}
%\end{figure}

\subsection{RQ2: Effect of number of clients on FOLTR-ES}
%To answer RQ1, we perform experiments on MSLR-WEB10K dataset with the same FOLtR-ES setup
%Reproducing FOLtR-ES on other datasets

For answering RQ2 we vary thee number of clients involved in FOLTR-ES; we investigate the values \{50, 1,000, 2,000\}. Kharitonov~\cite{kharitonov2019federated} used 2,000 in the original experiments, and the impact of the number . To be able to fairly compare results across number of clients, we fixed the total number of ranker updates to 2,000,000; we also set $B = 4$ and set the privatization parameter $p=0.9$. We perform these experiments on all three datasets considered in this paper, but we omit to report results for Yahoo due to space limitations. 


\begin{figure}[t]
	\centering
	\begin{subfigure}{1\textwidth}
		\includegraphics[width=15cm, height=3.5cm]{images/RQ2/mq2007_foltr_client_both_p0.9.png}
		\caption{Mean batch MaxRR for MQ2007.}
		\label{fig:mq2007-rq2}
	\end{subfigure}
	\begin{subfigure}{1\textwidth}
		\includegraphics[width=15cm, height=3.5cm]{images/RQ2/mslr10k_foltr_client_both_p0.9.png}
		\caption{Mean batch MaxRR for MSLR10k}
		\label{fig:mslr10k-rq2}
	\end{subfigure}
	\caption{Results for RQ2: performance of FOLTR-ES with respect to number of clients. \label{fig:RQ2}} 
\end{figure}


The results of these experiments are reported in Figure~\ref{fig:RQ2}, and they are mixed. For MQ2007, the number of clients have little effect on the neural ranker used in FOLTR-ES, although when informational clicks are provided this ranker is less stable, although often more effective, if very few clients (50) are used. Having just 50 clients, instead, severally hits the performance of the linear ranker, when compared with 1,000 or 2,000 clients. The findings on MSLR10k, however, are different. In this dataset, a smaller number of clients (50), is generally better than larger numbers, both for linear and neural ranker. An exception to this is when considering navigational clicks: in this case the linear ranker obtains by far thee best performance with a small number of clients, but the neural ranker obtains the worst performance. This suggest that the number of clients greatly affects FOLTR-ES: but trends are not consistent across click types and datasets. 


%To study the influence of number of clients, we perform experiments on MQ2007 and MSLR-WEB10K datasets. We vary the number of clients across \{50, 1000, 2000\} but set the fixed total updating times to 2000000 and set $B = 4$. We also set the privatization parameter $p$ across \{0.5, 0.9, 1.0\}.

%Our experiments show that little clients number will reduce the performance in the linear ranker. But for the neural ranker, the difference is minor.

%\begin{figure}[H]
%	\centering
%	\includegraphics[width=16cm, height=8cm]{v0_mq2007_foltr_clients_p09.png}
%	\caption{Mean batch MaxRR for MQ2007 with different client number}
%	\label{fig: mq2007clients}
%\end{figure}
%\begin{figure}[H]
%	\centering
%	\includegraphics[width=15cm, height=3.5cm]{mq2007_foltr_client_linear_p09.png}
%	\caption{Mean batch MaxRR for MQ2007 with different client number (linear ranker and $p = 0.9$)}
%	\label{fig: mq2007-rq2-0.9}
%\end{figure}

\subsection{RQ3: Comparing FOLTR-ES to state-of-the-art OLTR methods}
The original study of FOLTR-ES did not compared the method with non-federated OLTR approaches. To contextualise the performance of FOLTR-ES and to understand the trade-off between privacy and performance done when designing FOLTR-ES, we compare this method with the current state-of-the-art OLTR method, the Pairwise Differentiable Gradient Descent (PDGD)~\cite{oosterhuis2018differentiable}. For fair comparison, we set the privatization parameter $p=1$ (lowest privacy) and the number of clients to 2,000. In addition note that in normal OLTR settings, rankers are updated after each user interaction: however in FOLTR-ES, rankers are updated in small batches. For fair comparison, we adapt PDGD to be updated in batch too. Instead of updating the ranker after each interaction (batch size 1), we accumulate gradients computed on the same batch size as for FOLTR-ES. Specifically, with 2000 clients for FOLTR-ES, the batch size of each update is 8,000 iterations (4 x 2,000). We then compute the updated gradients for PDGD on 8,000 interactiond too. %Note, the number of updates after 2m user interaction now becomes 2m/8000 = 250.
We perform these experiments on all three datasets considered in this paper, but we omit to report results for Yahoo due to space limitations. 

\begin{figure}[t]
	\centering
	\begin{subfigure}{1\textwidth}
		\includegraphics[width=15cm, height=3.5cm]{images/RQ3n4/mq2007_foltr_PDGD_mrr_c2000_p1.0}
		\caption{Mean batch MaxRR for MQ2007.}
		\label{fig:mq2007-rq3}
	\end{subfigure}
	\begin{subfigure}{1\textwidth}
		\includegraphics[width=15cm, height=3.5cm]{images/RQ3n4/mslr10k_foltr_PDGD_mrr_c2000_p1.0.png}
		\caption{Mean batch MaxRR for MSLR10k}
		\label{fig:mslr10k-rq3}
	\end{subfigure}
	\caption{Results for RQ3: performance of FOLTR-ES and PDGD across datasets. \label{fig:RQ3}} 
\end{figure}

Results are shown in Figure~\ref{fig:RQ3}: regardless of linear or neural ranker, FOLTR-ES is less effective than PDGD. The gap in performance is greater in larger datasets like MSLR10k and Yahoo (not shown) than in the smaller MQ2007/2008. This gap becomes even bigger, especially for the first iterations, if the PDGD ranker was updated after each iteration (not shown here), rather than after a batch has been completed. This highlights that FOLTR-ES has the merit of being the first privacy preserving federated OLTR approach available; however, more work is needed to improve the performance of FOLTR based methods so as to close the gap between privacy-oriented approaches and centralise approaches that do not consider user privacy.

%In order to further study the ranking quality, especially users' experience in FOLtR-ES, we perform experiments on comparing FOLtR-ES with the current state-of-the-art OLTR methods. In this section, we choose Pairwise Differentiable Gradient Descent (PDGD) models as baselines. For a fair comparation, we set up the privacy probability $p = 1$ (lowest privacy) in FOLtR-ES. We perform experiments on MQ2007 and MSLR-WEB10K datasets with simulating 2000 clients.

%Based on the experiment results, we can see FOLtR-ES still lags behind OLTR methods in terms of the ranking performance. As the ranking quality is an essential metric for web search engines, future work can be extend to improving ranking performance and in the meantime, protecting users' privacy.

%\begin{figure}[H]
%	\centering
%	\includegraphics[width=16cm, height=8cm]{v0_mq2007_foltr_vs_pdgd_2000clients_p10.png}
%	\caption{Mean batch MaxRR for MQ2007 with 2000 clients and $p = 1$}
%	\label{fig: mq2007-v0-baseline}
%\end{figure}
%
%\begin{figure}[H]
%	\centering
%	\includegraphics[width=16cm, height=8cm]{v0_mslr_foltr_vs_pdgd_2000clients_p10.png}
%	\caption{Mean batch MaxRR for MSLR-WEB10K with 2000 clients and $p = 1$}
%	\label{fig: mslr-v0-baseline}
%\end{figure}
%
%\begin{figure}[H]
%	\centering
%	\includegraphics[width=15cm, height=3.5cm]{mq2007_foltr_PDGD_mrr_c2000_p10.png}
%	\caption{Mean batch MaxRR for MQ2007 with two FOLtR-ES ranker(2000 clients and $p = 1$) and PDGD baselines}
%	\label{fig: mq2007-rq3}
%\end{figure}


\subsection{RQ4: Extending FOLTR-ES evaluation to common OLTR practice}

In the original work and in the sections above, FOLTR-ES was evaluated using MaxRR computed with respect to the clicks performed by the simulated users (click models). This is an unusual evaluation for OLTR because: (1) usually nDCG@10 is used in place of MaxRR as metric, (2) nDCG is computed with respect to relevance labels, and not clicks, and on a withheld portion of the dataset, not on the interactions observed -- this is used to produce learning curves and is referred to as offline nDCG, (3) in addition online nDCG is measured from the relevance labels in the SERPs from which clicks are obtained, and either displayed as learning curves or accumulated throughout the sessions -- these values represent how OLTR has affected user experience. We then consider this more common evaluation of OLTR next, where we set the number of clients to 2,000 and experiment with $p=\{0.5, 0.9, 1.0\}$; we omit to report results for Yahoo due to space limitations. 


\begin{figure}[t]
	\centering
	\begin{subfigure}{1\textwidth}
		\includegraphics[width=15cm, height=3.5cm]{images/RQ3n4/mq2007_foltr_DCG_both_c2000_ps.png}
		\caption{Mean batch nDCG@10 for MQ2007.}
		\label{fig:mq2007-rq4}
	\end{subfigure}
	\begin{subfigure}{1\textwidth}
		\includegraphics[width=15cm, height=3.5cm]{images/RQ3n4/mslr10k_foltr_DCG_both_c2000_ps.png}
		\caption{Mean batch nDCG@10 for MSLR10k}
		\label{fig:mslr10k-rq4}
	\end{subfigure}
	\caption{Results for RQ4: performance of FOLTR-ES in terms of online nDCG@10 computed using relevance labels and the SERPs used for obtaining user iterations. \label{fig:RQ4}} 
\end{figure}


Results are reported in Figure~\ref{fig:RQ4}. It is interesting to compare these plots with those in Figure~\ref{fig:RQ1}, that relate to the unusual (for OLTR) evaluation setting used in the original FOLTR-ES work. By comparing the figures, we note that for MQ2007, FOLTR-ES can effectively learn rankers for perfect and navigational clicks. However, when the clicks become nosier (informational clicks), then FOLTR-ES learning is effective for the linear ranker but no learning occurs for the neural ranker: this is unlikely in the evaluation settings of the original work (Figure~\ref{fig:RQ1}). We note this finding repeating also for MSLR10k, but this time this affects both linear and neural rankers; we also note that the online performance in MSLR10k on navigational clicks are also quite unstable and exhibit little learning for specific values of $p$ and ranker type. These findings suggest that FOLTR-ES is yet far from being a solution that can be considered for use in practice, and more research is required for devising effective federated, privacy-aware OLTR techniques.



%To study the generalisation of FOLtR-ES to other common OLTR evaluation metrics, we use Discounted Cumulative Gain (DCG) to evaluate the users' experience in each interaction. We also compute NDCG@10 with the relevance labels to evaluate the central ranker with test data as we want to explore the stability of  performance in FOLtR-ES.

%In terms of nDCG evaluation, although FOLtR-ES achieves decent performance (with nDCG exceed 0.4 except for $Informational$ model), it still falls behind the PDGD baselines.

%\begin{figure}[H]
%	\centering
%	\includegraphics[width=15cm, height=3.5cm]{mq2007_foltr_PDGD_ndcg_c2000_p10.png}
%	\caption{Mean batch MaxRR for MQ2007 with(2000 clients and $p = 1$)}
%	\label{fig: mq2007-rq4}
%\end{figure}


\section{Related Work}

This section gives a brief overview of traditional LTR, OLTR and federated learning. \\
\\
1.offline learning to rank\\
2.online learning to rank\\
Main challenges in online learning to rank problem are: (a) noise and biases of user interactions, (b) non-continuous metrics for optimization.\\
3.federated machine learning\\




\section{Conclusions}

We set to explore 4 research questions. RQ1 aimed to investigate the generalisability of the original results obtained  by FOLTR-ES on the MQ2007/2008 dataset to other dataset used in current OLTR practice. Our reproducing experiments on MQ2007/2008 show the consistently similar findings with the original work. However, in terms of larger LTR datasets (MSLR-WEB10K and Yahoo datasets), the neural ranker is less effective than the linear ranker, especially on MSLR-WEB10K, which shows FOLtR-ES needs more data to achieve effective training on those datasets with larger number of features.
RQ2 aimed to investigate the effect varying the number of clients involved in FOLTR-ES has on the effectiveness of the method. We set the total times of interaction to a fix number and discover the ranking quality in terms of different number of clients in online simulation. Our experiments show that little clients number harm the performance in the linear ranker. But for the neural ranker, the difference is minor.
RQ3 aimed to compare FOLRT-ES with current OLTR state-of-the-art methods to understand the gap required to be paid for maintaining privacy. Our findings suggest that FOLtR-ES lags behind the OLTR baseline in terms of ranking performance
RQ4 aimed to investigate the generalisability of the original results obtained for FOLTR-ES to common evaluation practice in OLTR. \todo{Our findings suggest that ...}

\makeatletter
\renewcommand{\@biblabel}[1]{\hfill #1.}
\makeatother

\subsubsection*{Acknowledgements.} Shuyi Wang is sponsored by a China Scholarship Council (CSC) scholarship. Associate Professor Guido Zuccon is the recipient of an Australian Research Council DECRA Research Fellowship (DE180101579) and a Google Faculty Award. 

\bibliographystyle{splncs04}
\bibliography{ecir2021-foltr-reproducibility}

\end{document}

